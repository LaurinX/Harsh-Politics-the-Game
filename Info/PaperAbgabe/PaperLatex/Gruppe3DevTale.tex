\documentclass[conference]{IEEEtran}
\IEEEoverridecommandlockouts
% The preceding line is only needed to identify funding in the first footnote. If that is unneeded, please comment it out.
\usepackage{cite}
\usepackage{amsmath,amssymb,amsfonts}
\usepackage{algorithmic}
\usepackage{graphicx}
\usepackage{textcomp}
\usepackage{xcolor}
\usepackage{pgfplots}
\def\BibTeX{{\rm B\kern-.05em{\sc i\kern-.025em b}\kern-.08em
    T\kern-.1667em\lower.7ex\hbox{E}\kern-.125emX}}
\begin{document}

\title{Phasen, Techniken und Strategien der
Softwareentwicklung an Hand eines
Kursprojekts im Modul SWE-II*\\
{\footnotesize \textsuperscript{*}Thema Vorgabe laut Checkliste zum Kursprojekt im Modul Software-Engineering-II}
}

\author{\IEEEauthorblockN{Fabian Schiemann}
\IEEEauthorblockA{\textit{Student at Computer Science Dept} \\
\textit{Faculty of Cooperative Studies}\\
\textit{Berlin School of Economics and Law, Germany}\\
s\_schiemann21@stud.hwr-berlin.de}
\and
\IEEEauthorblockN{Oskar Hugo Thaute}
\IEEEauthorblockA{\textit{Student at Computer Science Dept} \\
\textit{Faculty of Cooperative Studies}\\
\textit{Berlin School of Economics and Law, Germany}\\
s\_thaute21@stud.hwr-berlin.de}
\and
\IEEEauthorblockN{Sharleen Schnelle}
\IEEEauthorblockA{\textit{Student at Computer Science Dept} \\
\textit{Faculty of Cooperative Studies}\\
\textit{Berlin School of Economics and Law, Germany}\\
s\_schnelle21@stud.hwr-berlin.de}
\and
\IEEEauthorblockN{Noah Reckhard}
\IEEEauthorblockA{\textit{Student at Computer Science Dept} \\
\textit{Faculty of Cooperative Studies}\\
\textit{Berlin School of Economics and Law, Germany}\\
s\_reckhard21@stud.hwr-berlin.de}
\and
\IEEEauthorblockN{Laurin Pausch}
\IEEEauthorblockA{\textit{Student at Computer Science Dept} \\
\textit{Faculty of Cooperative Studies}\\
\textit{Berlin School of Economics and Law, Germany}\\
s\_pausch21@stud.hwr-berlin.de}
\and
\IEEEauthorblockN{Minh Hai Nguyen}
\IEEEauthorblockA{\textit{Student at Computer Science Dept} \\
\textit{Faculty of Cooperative Studies}\\
\textit{Berlin School of Economics and Law, Germany}\\
s\_nguyenmi21@stud.hwr-berlin.de}
}

\maketitle

\begin{abstract}
Die Theorie der Softwareentwicklung bietet zahlreiche strukturierte Informationen und Vorstellungen von Form und Struktur eines Softwareprojektes. Die Realität innerhalb eines solchen Projektes kann davon jedoch stark variieren. Sei es der Vorgesetzte in einem alteingesessenen Familienunternehmen, der Scrum-Sprints verplant, oder der rüstige Teamleiter einer großen Abteilung, der alle seine Projekte in PASCAL geschrieben haben möchte. Dieser Artikel präsentiert eine solche Variation und zeigt, wie sich eine Gruppe von sechs Studenten mit hohen Ambitionen und frisch erlangtem Wissen einem solchen Softwareprojekt widmet. Generationsbedingt wird im Folgenden die Entwicklung eines Computerspiels mit dem Namen "Harsh Politics" vorgestellt. Ziel dieses Artikels ist die Darlegung und Erläuterung spezieller Methoden und Strategien aus der Softwareentwicklung, anhand der Entwicklung eines Spieles.
\end{abstract}

\section{Hinweise}
Aus Gründen der besseren Lesbarkeit wird im Text verallgemeinernd die männliche Form verwendet.
Diese Formulierungen umfassen gleichermaßen weibliche und männliche Personen.
\section{Einleitung / Motivation / Einführung / Hintergründe}
\input{intro.txt}

\section{Anforderungen}
\input{requirements.txt}

\section{Organisation}
\input{orga.txt}

\section{Umsetzung}
\input{umsetzung.txt}

\section{Evaluation}
\input{eval.txt}

\section{Fazit}
\input{fazit.txt}


\begin{thebibliography}{00}
\bibitem{b1} Booch, G. (2018). The History of Software Engineering. IEEE Software, 35(5), 108–114. https://doi.org/10.1109/ms.2018.3571234
\bibitem{b2} Sommerville, I. (2011). Software Engineering. Addison Wesley Longman.
\bibitem{b3} DRY vs KISS – Clean Code Prinzipien. (2018, 25. Mai). genericde. Abgerufen am 20. März 2023, von https://www.generic.de/blog/dry-vs-kiss-clean-code-prinzipien
\bibitem{b4} Oloruntoba, S. (2021, 19. Februar). SOLID: Die ersten 5 Prinzipien des objektorientierten Designs. DigitalOcean. Abgerufen am 20. März 2023, von https://www.digitalocean.com/community/conceptual-articles/s-o-l-i-d-the-first-five-principles-of-object-oriented-design-de
\bibitem{b5} Was ist YAGNI? | Softwareentwicklung | PI-Lexikon. (2023, 23. März). pi-informatik. Abgerufen am 24. März 2023, von https://www.pi-informatik.berlin/pi-lexikon/softwareentwicklung/was-ist-yagni/
\bibitem{b6} Folien Lara Maria Stricker HWR-Berlin

\end{thebibliography}
\vspace{12pt}

\end{document}
